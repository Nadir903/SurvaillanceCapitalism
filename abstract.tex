
\begin{abstract}
Surveillance capitalism has raised significant ethical concerns regarding the collection and utilization of personal data by tech giants like Google.
This paper explores these ethical implications through various ethical frameworks, including deontological ethics, consequentialism, and hedonism.
The analysis reveals critical issues related to privacy, autonomy, and fairness, and proposes potential solutions to mitigate these concerns.
By examining case studies of Google's practices, including data breaches and unauthorized data collection, the paper highlights the need for stricter regulations and ethical data practices to protect user rights in the digital age.
Ensuring transparency, informed consent, and robust data protection measures are essential for balancing the benefits of data-driven innovation with respect for individual rights.
\end{abstract}
