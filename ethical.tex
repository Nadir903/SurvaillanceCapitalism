\section{Ethical Frameworks}\label{sec:ethical-frameworks}

\subsection{Deontological Perspective}\label{subsec:deontological-perspective}
Google's data practices often violate principles of autonomy and consent.
Users are typically not fully informed about the extent of data collection, which infringes on their autonomy and right to privacy.
According to Kantian ethics, this is inherently unethical as it fails to respect the intrinsic worth of individuals, violating the principle of treating people as ends in themselves.
Additionally, Kant's principle of the categorical imperative suggests that actions are only morally right if they can be universalized.
The practice of collecting personal data without consent cannot be ethically justified as a universal law because it would lead to a society where privacy is routinely violated, undermining trust and freedom.
\subsection{Contractualist Perspective}\label{subsec:contractualist-perspective}
The implicit agreement between Google and its users is often unfair, with users not adequately informed about how their data will be used.
This lack of transparency undermines trust and the social contract.
Contractualists would argue that data practices should be based on fair agreements that all parties, including users, would consent to under reasonable conditions.
Furthermore, the principle of reciprocity is violated as users provide valuable data without receiving fair compensation or benefits in return.
The imbalance of power and information between Google and its users creates an unethical dynamic where users are exploited for their data.
\subsection{Consequentialist Perspective}\label{subsec:consequentialist-perspective}
Google's services provide significant benefits, including enhanced user experience and economic growth.
However, these benefits are countered by privacy invasions, potential for data breaches, and manipulation of user behavior.
From a utilitarian perspective, the overall consequences of surveillance capitalism might not justify the means if the negative impacts on individual and societal well-being are substantial.
The long-term consequences of widespread data collection include the erosion of trust in digital services, increased anxiety and mental health issues related to privacy concerns, and potential misuse of data by malicious actors.
These negative outcomes must be weighed against the economic and convenience benefits provided by personalized services.
\subsection{Hedonistic Perspective}\label{subsec:hedonistic-perspective}
While Google's services increase user pleasure through convenience and personalization, the anxiety and distress caused by privacy concerns and data misuse result in significant net pain.
A hedonist would weigh the pleasure and pain resulting from surveillance capitalism to determine its ethicality.
If the negative impacts on well-being are substantial, the practice might be deemed unethical.
The subjective nature of pleasure and pain also highlights the diversity of user experiences.
While some users may greatly benefit from personalized services, others may suffer from the constant surveillance and lack of privacy, leading to a more complex ethical evaluation.