
\begin{abstract}
    Surveillance Capitalism refers to the economic system in which personal data is captured through surveillance and then monetized.
    This term was first coined by Shoshana Zuboff in her paper with the same name.
    There she explains how this practice has become increasingly prevalent with the rise of giants companies like Google, which collect gigantic amount of user data in order to optimize services and display targeted advertising.
    This paper is going to explore the ethical implications of such practices, just as Google's data practices and collectors, through various ethical lenses, including deontological ethics, consequentialism, and hedonism.
    This analysis also reveals issues and concerning topics related to privacy, autonomy and fairness.Furthermore, this paper advocates for stringent regulatory oversight, corporate accountability and increased citizenry awareness to protect user rights in the digital era.
    Future research should focus on developing frameworks and policies that balance the benefits of data-driven innovation with respect for individual privacy and autonomy.
\end{abstract}

