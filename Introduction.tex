\section{Introduction}\label{sec:introduction}
Surveillance capitalism refers to the economic system where personal data is monetized captured through digital surveillance.
The origins of surveillance capitalism can be traced back to the early 2000s, where internet giants like Google and Facebook have risen.
Companies like those were the first to see the huge value of user data, because it was key to predict their behavior and build targeted features and services, as well as ads for income.
Then this practice began to take a more concrete shape as these companies developed sophisticates methods and tools to collect, analyze and utilize data of their user.
It can be discussed that this practice has also advantages, such as improvement of user experience and service personalization but, it can also quickly become apparent that this data could be leveraged for targeted advertising, leading to significant revenue streams.

The early era of internet was primarily dominated by a collection of static websites and simple services, but as the internet evolved, so did the business models of the companies that dominated it.
We are going to take Google as an example to explain how surveillance capitalism took its first steps.
Google was founded in 1998, and it was initially focused on creating the most advanced and quicker search engine of its time.
However, the company soon realized that the collected data through user searches could have a real useful porpoise, since at the time other search engines just discarded that information.
These saved queries could be used to predict user interest and sell this data to other companies and as well as deliver highly targeted advertisements .
This realization marked the beginning of surveillance capitalism, a trend which would only grow in the following years.

With the introduction of AdWords (now Google Ads) in 2000 and AdSense in 2003 by Google, a new revolution for online advertising took place.
These platforms used stored user data to serve targeted ads, which significantly increased the effectiveness and relevance of advertisement the users were displayed.
This happened because users were now getting ads that matched with their necessities and curiosity.
Therefore, companies like Facebook, which was founded in 2004, took a similar business model, not only because the profitability of data-driven advertising was demonstrated, but also due to the unawareness of users, who kept feeding the collector machines developed by these companies.
The fact that Facebook rapidly expanded its user base and began collecting vast amounts of data on user interactions shows how effective this business model was.
This collected data the foundation for its advertising platform, which allowed advertisers to target users based on their interests, behaviors, and demographics.

With the evolution of surveillance capitalism, new study fields as data analytics have been developed.
In addition to that, the proliferation of mobile devices, other social media platforms and IoT devices has led to a huge boom in the amount of data generated by users.
In response to this, companies have developed more advanced and sophisticated algorithms which analyzes this user data and extract valuable insights, which can either be used to profile those users and send them high targeted ads and content based on their behavior, or can be sold to other companies with similar practices.

Today, surveillance capitalism counts as the dominant business model for many of the world's largest and more powerful technology companies.
Companies as Google, Meta and Amazon generate significant portions of their revenue through data-driven advertising.
Thus, these companies have built vast and user-friendly ecosystems which collect data from a wide range of hardware and software interactions, such as physical devices or user behavior in certain social media.
For that very reason, surveillance capitalism has got a huge impact on society.
On one hand, it has helped with the development of highly personalized services, which enhances user experience.
But on the other hand, it has raised significant ethical and privacy concerns leading up to following questions: How private is our life and habits as well as to what extent are we owners of our information.

Therefore, is important to question how well regulation and oversight to protect user privacy and ensure accountability are managed and executed by these companies.
The ethical implications of surveillance capitalism are complex and multifaceted, that is why it is important to raise questions about consent, autonomy, and transparency.
Furthermore, the use of data to influence behavior raises questions about manipulation and autonomy.

For all those reasons and by analyzing specific cases studies applying them through various ethical frameworks, the paper aims to analyze the critical ethical issues and propose potential solutions to ensure ethical data practices in the digital age.
\noindent
