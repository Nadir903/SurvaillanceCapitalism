\section{Ethical Frameworks}\label{sec:ethical-frameworks}
The ethical implications of surveillance capitalism are going to be explored in this section through three primary ethical lenses: Deontology, Consequentialism, and Hedonism.
Each ethical lens provides us a unique approach about the morality of the practices of many mentioned firms and what should be done, in order to stick to that moral.

\subsection{Deontological Perspective}\label{subsec:deontological-perspective}
Deontology focuses on the rightness or wrongness of actions according to rules established in a society rather than their consequences and therefore, follows premises of the contractualism too.
From a deontological point of view, data practices like those from Google and Facebook violate principles of informed consent.
This is because users are not fully aware of the extent of data these companies collect from them.
Likewise, the terms and conditions when using their platform are written in a vague and very extended way, what leads the users not to read them and accept everything they are asked to.
We could classify this as a form of misinformation, which violates principles of informed consent and the right to privacy the user should be granted with.

According to Kantian ethics, this is inherently unethical as it fails to respect the intrinsic worth of individuals\cite{Kant1998_Groundwork}.
This violates the principle of treating people as humans and transforming them as ends or assets rather than autonomous beings.

Kant’s categorical imperative also suggests that actions are only morally right if they are universal, holding for every person.
That is the reason why the practice of collecting personal data without explicit consent cannot be ethically justified as a universal law.
It wouldn't lead to a society where privacy is routinely violated, but also to the eradication of trust and individual worth.

This deontological perspective emphasizes the importance of respecting individuals' rights and autonomy over the private benefits these companies would get by treating them just as a source of data and income.
Including to that, the principle of transparency is crucial for this perspective.
User should be fully aware about how their data is collect and what is being done with that.
They also must consent freely and informed to use what these companies offer without scams or manipulation.

Additionally, deontological ethics criticize the exploitation of human information while practicing the surveillance capitalism.
The gains for both parties should be equal and fair.
Practices like user's data being harvested and monetized without their full understanding or fair compensation just violates the deontological principle of treating individuals with dignity and respect.


\subsection{Consequentialist Perspective}\label{subsec:consequentialist-perspective}
Consequentialism evaluates the morality of actions based on their outcomes and overall consequences, it is where the famous phrase \("\)The end justifies the means\("\) comes from.
Therefore, from a consequentialist point of view, the practices of firms like Google provides significant benefits.
These benefits, for both these companies and the final users, are user experience, personalized services, and economic growth.
these benefits must be weighed against the potential negative consequences to be justified.
This negative consequences are, as previously mentioned, invasions of privacy, data breaches, and the manipulation of user behavior.

For example, the utilitarianism, a form of consequentialism, postulates that every action is morally right if and only if it achieves the greatest happiness for the greatest number\cite{Mill1863_Utilitarianism}.
Therefore, we could say that personalized services and targeted advertisements are right practices because they might improve user convenience and satisfaction.
But it important to analyse the long-term consequences of this widespread data collection.
These consequences can be erosion of trust of the users against the companies, increased anxiety and mental health issues related to privacy concerns, as well as potential misuse of data by malicious parties.
As a result, if the negative impacts on individual and societal well-being are substantial, then the practice of surveillance capitalism is not justified by a consequentialist point of view.

With that previous reasoning, it can be concluded that the malicious use of big data analytics to , for example, influence political outcomes, as seen in the Cambridge Analytica scandal, demonstrates how data can be used to manipulate public opinion.
For that reason, it is also important to study the long term consequences of data-driven services, particularly for such scenarios where harm may outweigh the benefits.


\subsection{Hedonistic Perspective}\label{subsec:hedonistic-perspective}
Hedonism considers the pursuit of pleasure and the avoidance of pain as the most important goals\cite{Crisp2006_HedonismReconsidered}.
Therefore, it can be said that from a hedonistic perspective, the collection of data to increase user pleasure through convenience and personalization is ethicality correct.
It is however contradicted if this collection means a violation of privacy for the users, increasing anxiety and distress for the potential for data misuse, resulting in pain.

A hedonist would assess how justified the practices of surveillance capitalism by weighing the pleasure and pain it generates.
But this is partly difficult to analyse, since not all actions provokes pain or pleasure for all individuals alike.
While some user may, for example, enjoy in a significant way the personalizes services data-driven companies provide, other may experience considerable discomfort to the fact that they are constantly being monitored and that the content they visualize is highly targeted to them, meaning a sort of manipulation.
If this negative impacts on well-being outweigh the benefits, surveillance capitalism could be deemed unethical from a hedonistic point of view.

But as said before, the subjective nature of pleasure and pain highlights the diversity of user experiences.
Hedonism therefore minimizes the need to consider individual experiences when evaluating the ethics of data practices.
This is because it is recognized that the same practice can produce different outcomes for different people and there is for instance no universal truth.
One example is that, while targeted advertisements might enhance shopping experiences for some users, others might find them as intrusive and unsettling experiences.

Moreover, the constant monitoring and collection of data can lead to a pervasive sense of being watched.
This could cause in some individual psychological stress and even if this doesn't hold for everybody, it doesn't take away from the fact that such concerns just reduce overall happiness.

For that reason, unless this negative emotional impact is balanced against the positive experiences derived from data-driven personalization, it can not be held ethicality correct from a hedonist perspective.