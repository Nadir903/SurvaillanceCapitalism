\section{Case Studies}\label{sec:case-studies}

Has we have seen in the previous section of this paper, there are many scenarios where it is shown that there are benefits as efficiency and enjoyment.
But they come at the cost of demanding the user data as an essential part of the architecture of these platforms, whose business model is the surveillance capitalism.
For that reason, this paper is going to analyse real case scenarios, where this practice was exposed to the public and what was their reaction.

\subsection{Scandal of Cambridge Analytica}\label{subsec:scandal-of-cambridge-analytica}
One of the most prominent examples of how personal information was misused for political purposes occurred with the case of Cambridge Analytica, a political consulting firm.
Even though it started in 2016, it was first revealed in the year 2018 that the personal data of millions of Facebook users was harvested by Cambridge Analytica\cite{Cadwalladr2018_CambridgeAnalytica} without their consent.
This data was used to build psychological and demographic profiles to target individuals with personalized political propaganda during the president US election in 2016\cite{Rosenberg2018_ExploitedData} and the Brexit referendum.
It was then exposed through this scandal how many ethical and legal issues surrounding data privacy was happening without public knowledge and that political manipulation was a reality not many were aware of.
The scandal exposed significant ethical and legal issues surrounding data privacy and the use of personal data for political manipulation.
Facebook faced then intense regulatory investigations as well as numerous trials for its role in the data breach and its failure to protect user privacy.
The incident highlighted the lack of transparency in facebook's platform engines and the negligent oversight in data collection practices.
For that very reason, many facebook users deleted their accounts and the platform lost its popularity, particularly the facebook social media app.
This was not only because of the lack of integrity and transparency, but also due to the growing mistrust and concerns about the influence and manipulation of social media platforms on democratic processes.
The ethical implications of the Cambridge Analytica scandal are many.
On the one hand, the social network is presented as a free way to connect with other people and to subscribe to many pages where this people shares their interest.
It is about connecting people, as their slogan always said, but on the other hand, it was shown that robust data protection laws and greater accountability for companies that collect and use personal data weren't priority until cases like this happened.
The incident also emphasizes the importance of informed consent and the ethical responsibilities organization should follow, if their business model handles sensitive information about users as well as if they can be used for political or ideological manipulation.

\subsection{Incognito Mode Tracking Lawsuit}\label{subsec:incognito-mode-tracking-lawsuit}
In June 2020, a class-action lawsuit was filed against Google and its incognito mode by web browsing\cite{Google2020_incognitoData}.
These accusations were alleging that the tech-company was tracking the internet activity and behavior of its users, even when they were using Incognito mode.
It is important to add that this feature was designed and advertised for private browsing.
The lawsuit also claimed that Google was harvesting this user information through various tools and means, such as Google Analytics, Google Ad Manager, and website plug-ins.
Those features of the search engine were operating in the incognito modus too without knowledge of users and the plaintiffs argued that this was violating federal wiretap laws and California laws for privacyö
With this lawsuit, many people became aware about significant ethical concerns about privacy and deception.
Incognito mode was perceived by many users as a way to browse the internet without leaving a tack of their activity, but this promise was contradicted by the these practices now shown to the public, undermining trust among the users.
Users who relied on Incognito mode to protect their privacy felt deceived and exposed.
The ethical implications of tracking users in Incognito mode are considerable.
Not only because the deceiving promise of privacy, but also because it breaches the ethical principles of honesty and respect for user autonomy.
Trust is a fundamental component of the user-technology relationship, but trust is eroded when both parties aren't transparent.
Furthermore, the lawsuit highlighted issues related to informed consent, because users were not fully informed about the large amount collected of data in Incognito mode.
This collection without explicit user consent violates privacy rights and can lead to misuse of personal information.
But this case also brought attention to the broader implications of data collection practices by tech companies.
For instance, more people were aware of how important it is the need for greater transparency and accountability in how user data is collected and, more important, if it is with consent.
For that the lawsuit called for stricter regulations and enforcement to protect user privacy and ensure that companies hold their privacy promises.

\subsection{Apple and the San Bernardino Case}\label{subsec:apple-and-the-san-bernardino-case}
Apple was involved in a legal battle with the FBI for unlocking a device manufactured by them in 2016\cite{Benner2016_AppleFightsOrder}.
The suite was about whether Apple should help unlock the iPhone of Syed Rizwan Farook.
This person was one of the perpetrators of the San Bernardino terrorist attack in December 2015, attack resulted in 14 deaths and 22 injuries.
Due to the magnitude of this event, federal investigation was involved.

Apple's assistance in bypassing the iPhone's security features was asked by the FBI in order to access potential evidence, that could be used against Rizwan for the case related to the terrorist attack.
The FBI wanted Apple to create was a special version of iOS where certain security features would be disabled, such as the auto-erase function after ten incorrect password attempts.
Actually with this modification, the FBI would be able to use brute force to get the iPhone's passcode and access its content.
Apple refused the request, citing concerns over user privacy and security.
The company argued that creating a backdoor to the iPhone would set a dangerous precedent which could be exploited by many malicious people, hackers and private or government agencies, which could potentially compromise the security of all iPhone users.

The ethical implications of this case were again significant.
On one hand, there was a good intention in investigating a serious and dangerous terrorist, which could prevent attacks future incidents.
But on the other hand, creating a backdoor means severe risks to user privacy and data security, specially from a company known for its dedication to privacy.
Therefore, Apple stood for protecting the security and privacy of its users, even when that implied that the company should give up helping the society by unlocking this single device for law enforcement purposes.
The case started also a nationwide concern on privacy and security as well as how justified is the government authority in accessing private data.
Important questions like those and about the balance between national security and individual privacy rights were raised, even though the FBI found an alternative method to unlock the iPhone without Apple's assistance.

Here we could see of tech giants handled collected user data within their ecosystems.
While Facebook and Google have faced significant criticism and regulatory scrutiny over their data practices, Apple has taken a more privacy-focused stance.
The contrast between these companies remarks different approaches to how user data should be handled and privacy concerns really matter to these companies, with Apple emerging as a proponent of robust data protection and user privacy.
But this does not have to be necessarily the best approach, since threats to our society should be avoided.
Therefore, in the next section of the paper, these scenarios and dilemmas are going to be discussed through many ethical lenses, reminding the importance of privacy in the digital age but also what is the best for final users.
