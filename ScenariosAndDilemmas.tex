\section{Scenarios and Dilemmas}\label{sec:scenarios-and-dilemmas}
The spread of practices surveillance capitalism, mostly by gathering vast quantity of user data without their knowledge and consent, has steered to numerous ethical dilemmas.
Before going to real life case scenarios, this section is going to explore various general cases where the benefits of data-driven technologies clash with ethical considerations.
These scenarios mark the challenges of finding a middle point between innovation, user convenience, and privacy.

\subsection{Personalised Advertising}\label{subsec:personalised-advertising}

\textbf{Dilemma}: Let's take the case where a social media platform uses detailed user data to deliver highly targeted ads, improving ad effectiveness and enhanced user experience.

\textbf{Issue}: Personalised advertising involves collecting vast amounts of user data through a large variety smart devices, like smartphones and home assistants, which gather information such as voice commands, usage patterns, location data, and biometric information.
While these targeted ads can enhance user experience by showing relevant content for the users, they may not be fully aware of the amount of data collection or how their information is used, stored, or shared.
This lack of transparency raises concerns about informed consent and user autonomy.
The ethical dilemma here is whether the benefits of improved user experience and ad effectiveness justify the potential invasion of privacy and the risk of data misuse.

\subsection{Smart Home Devices}\label{subsec:smart-home-devices}

\textbf{Dilemma}: Smart home devices, such as smart speakers and security cameras, collect data by use to optimize home automation and improve user convenience.

\textbf{Issue}: These smart home devices practice an active monitoring of activities within the home in a continuous way.
This constant surveillance means, there is a significant privacy concern due to the full unawareness of users about the implications of having their personal lives recorded and analyzed by third parties.
The collected data can include audio recordings, video footage, and also patterns of daily activities, studied and analyzed by the algorithms these devices processes the data with.
It is true that these devices offer enhanced convenience and security, but they also mean a huge risk related to unauthorized access, data breaches, and the potential misuse of personal information.
Therefore, this ethical issue is going to persist until a balance between the convenience provided by smart home devices and the need to protect user privacy and maintain trust is sheltered.

\subsection{Health Data Collection}\label{subsec:health-data-collection}

\textbf{Dilemma}: A fitness app give users the opportunity to collect their health and activity data to provide them personalised health insights and recommendations.

\textbf{Issue}: Fitness and health apps track a large set of personal health metrics, such as physical activity, sport routines, heart rate and sleep patterns.
But while users benefit from personalised health advice or reminders to do fitness activities, the detailed monitoring of these health metrics can be really intrusive or not consented and there is the risk that sensitive health data could be accessed or used inappropriately.
There issues arise therefore concerns about data security, confidentiality and the exposure of private users based on their health information.
For that reason users may feel uncomfortable with the level of surveillance and the risk that their health information could be shared with third parties without their explicit consent.

\subsection{Free Social Media Platform}\label{subsec:free-social-media-platform}

\textbf{Dilemma}: A platform uses collected data of its users to personalise their feeds and enhance their interaction and experience, showing them content they probably like and ads tailored to their interests.

\textbf{Issue}: Social media platforms use algorithms to personalise content and ads based on previous interactions, influencing what users see and potentially shaping their opinions and behavior.
While personalization of these platforms can enhance user engagement and achieve their satisfaction, it also raises concerns about autonomy and manipulation.
This is because users might be unknowingly guided by recommendations or publications selected by a fine-tuned algorithm, creating a close space where they are only exposed to information that reinforces their existing beliefs.
This can lead to polarization and a lack of exposure to diverse perspectives.
That is the problem of this situation, that these algorithmic processes are not transparent and can impact on user autonomy.
Additionally, the concentration of power in the hands of platform operators raises concerns about accountability and the potential for abuse, mainly for using this power for political and ideological indoctrination.
